%% Generated by Sphinx.
\def\sphinxdocclass{report}
\documentclass[a4paper,11pt,english]{sphinxmanual}
\ifdefined\pdfpxdimen
   \let\sphinxpxdimen\pdfpxdimen\else\newdimen\sphinxpxdimen
\fi \sphinxpxdimen=.75bp\relax
\ifdefined\pdfimageresolution
    \pdfimageresolution= \numexpr \dimexpr1in\relax/\sphinxpxdimen\relax
\fi
%% let collapsible pdf bookmarks panel have high depth per default
\PassOptionsToPackage{bookmarksdepth=5}{hyperref}

\PassOptionsToPackage{warn}{textcomp}
\usepackage[utf8]{inputenc}
\ifdefined\DeclareUnicodeCharacter
% support both utf8 and utf8x syntaxes
  \ifdefined\DeclareUnicodeCharacterAsOptional
    \def\sphinxDUC#1{\DeclareUnicodeCharacter{"#1}}
  \else
    \let\sphinxDUC\DeclareUnicodeCharacter
  \fi
  \sphinxDUC{00A0}{\nobreakspace}
  \sphinxDUC{2500}{\sphinxunichar{2500}}
  \sphinxDUC{2502}{\sphinxunichar{2502}}
  \sphinxDUC{2514}{\sphinxunichar{2514}}
  \sphinxDUC{251C}{\sphinxunichar{251C}}
  \sphinxDUC{2572}{\textbackslash}
\fi
\usepackage{cmap}
\usepackage[T1]{fontenc}
\usepackage{amsmath,amssymb,amstext}
\usepackage{babel}



\usepackage{tgtermes}
\usepackage{tgheros}
\renewcommand{\ttdefault}{txtt}



\usepackage[Bjarne]{fncychap}
\usepackage{sphinx}

\fvset{fontsize=auto}
\usepackage{geometry}


% Include hyperref last.
\usepackage{hyperref}
% Fix anchor placement for figures with captions.
\usepackage{hypcap}% it must be loaded after hyperref.
% Set up styles of URL: it should be placed after hyperref.
\urlstyle{same}


\usepackage{sphinxmessages}
\setcounter{tocdepth}{2}



\title{FEO Global Documentation}
\date{Feb 21, 2023}
\release{unknown}
\author{TransitionZero}
\newcommand{\sphinxlogo}{\vbox{}}
\renewcommand{\releasename}{Release}
\makeindex
\begin{document}

\pagestyle{empty}
\sphinxmaketitle
\pagestyle{plain}
\sphinxtableofcontents
\pagestyle{normal}
\phantomsection\label{\detokenize{index::doc}}



\chapter{Contents}
\label{\detokenize{index:contents}}
\sphinxstepscope


\section{Indonesia}
\label{\detokenize{1_indonesia:indonesia}}\label{\detokenize{1_indonesia::doc}}
\sphinxAtStartPar
The first application of FEO Global is the development of an openly available
electricity systems model for Indonesia. This model is used to explore
transition pathways to a net\sphinxhyphen{}zero electricity system. As with all energy system
models, the inputs include a range of datasets and assumptions (e.g. technolgy
cost projections, discount rates). All of these inputs are described here in
order to allow for the model to be reviewed, re\sphinxhyphen{}run, and re\sphinxhyphen{}purposed.


\subsection{Model scope}
\label{\detokenize{1_indonesia:model-scope}}
\sphinxAtStartPar
The model aims to represent the electricity system of Indonesia as accurately as
possible, subject to constraints on data and computation time. The main aspects
that improve the accuracy of the model’s representation of Indonesia’s
electricity system are its spatial and temporal resolution.


\subsubsection{Spatial resolution}
\label{\detokenize{1_indonesia:spatial-resolution}}
\sphinxAtStartPar
The model represents all 34 provinces of Indonesia across 7 regions \sphinxhyphen{} shown in
the table and map below \sphinxhyphen{} as individual nodes.


\begin{savenotes}\sphinxatlongtablestart\begin{longtable}[c]{|\X{50}{200}|\X{50}{200}|\X{50}{200}|\X{50}{200}|}
\sphinxthelongtablecaptionisattop
\caption{Provinces of Indonesia\strut}\label{\detokenize{1_indonesia:id1}}\\*[\sphinxlongtablecapskipadjust]
\hline
\sphinxstyletheadfamily 
\sphinxAtStartPar
Region
&\sphinxstyletheadfamily 
\sphinxAtStartPar
Province
&\sphinxstyletheadfamily 
\sphinxAtStartPar
Province (English)
&\sphinxstyletheadfamily 
\sphinxAtStartPar
Model code
\\
\hline
\endfirsthead

\multicolumn{4}{c}%
{\makebox[0pt]{\sphinxtablecontinued{\tablename\ \thetable{} \textendash{} continued from previous page}}}\\
\hline
\sphinxstyletheadfamily 
\sphinxAtStartPar
Region
&\sphinxstyletheadfamily 
\sphinxAtStartPar
Province
&\sphinxstyletheadfamily 
\sphinxAtStartPar
Province (English)
&\sphinxstyletheadfamily 
\sphinxAtStartPar
Model code
\\
\hline
\endhead

\hline
\multicolumn{4}{r}{\makebox[0pt][r]{\sphinxtablecontinued{continues on next page}}}\\
\endfoot

\endlastfoot

\sphinxAtStartPar
Jawa
&
\sphinxAtStartPar
Banten
&
\sphinxAtStartPar
Banten
&
\sphinxAtStartPar
IDNBT
\\
\hline
\sphinxAtStartPar
Jawa
&
\sphinxAtStartPar
Jakarta Raya
&
\sphinxAtStartPar
Jakarta
&
\sphinxAtStartPar
IDNJK
\\
\hline
\sphinxAtStartPar
Jawa
&
\sphinxAtStartPar
Jawa Barat
&
\sphinxAtStartPar
West Java
&
\sphinxAtStartPar
IDNJB
\\
\hline
\sphinxAtStartPar
Jawa
&
\sphinxAtStartPar
Jawa Tengah
&
\sphinxAtStartPar
Central Java
&
\sphinxAtStartPar
IDNJT
\\
\hline
\sphinxAtStartPar
Jawa
&
\sphinxAtStartPar
Jawa Timur
&
\sphinxAtStartPar
East Indonesia
&
\sphinxAtStartPar
IDNJI
\\
\hline
\sphinxAtStartPar
Jawa
&
\sphinxAtStartPar
Yogyakarta
&
\sphinxAtStartPar
Yogyakarta
&
\sphinxAtStartPar
IDNYO
\\
\hline
\sphinxAtStartPar
Kalimantan
&
\sphinxAtStartPar
Kalimantan Barat
&
\sphinxAtStartPar
West Kalimantan
&
\sphinxAtStartPar
IDNKB
\\
\hline
\sphinxAtStartPar
Kalimantan
&
\sphinxAtStartPar
Kalimantan Selatan
&
\sphinxAtStartPar
South Kalimantan
&
\sphinxAtStartPar
IDNKS
\\
\hline
\sphinxAtStartPar
Kalimantan
&
\sphinxAtStartPar
Kalimantan Tengah
&
\sphinxAtStartPar
Central Kalimantan
&
\sphinxAtStartPar
IDNKT
\\
\hline
\sphinxAtStartPar
Kalimantan
&
\sphinxAtStartPar
Kalimantan Timur
&
\sphinxAtStartPar
East Kalimantan
&
\sphinxAtStartPar
IDNKI
\\
\hline
\sphinxAtStartPar
Kalimantan
&
\sphinxAtStartPar
Kalimantan Utara
&
\sphinxAtStartPar
North Kalimantan
&
\sphinxAtStartPar
IDNKU
\\
\hline
\sphinxAtStartPar
Maluku
&
\sphinxAtStartPar
Maluku
&
\sphinxAtStartPar
Maluku
&
\sphinxAtStartPar
IDNMA
\\
\hline
\sphinxAtStartPar
Maluku
&
\sphinxAtStartPar
Maluku Utara
&
\sphinxAtStartPar
North Maluku
&
\sphinxAtStartPar
IDNMU
\\
\hline
\sphinxAtStartPar
Nusa Tenggara
&
\sphinxAtStartPar
Bali
&
\sphinxAtStartPar
Bali
&
\sphinxAtStartPar
IDNBA
\\
\hline
\sphinxAtStartPar
Nusa Tenggara
&
\sphinxAtStartPar
Nusa Tenggara Barat
&
\sphinxAtStartPar
West Nusa Tenggara
&
\sphinxAtStartPar
IDNNB
\\
\hline
\sphinxAtStartPar
Nusa Tenggara
&
\sphinxAtStartPar
Nusa Tenggara Timur
&
\sphinxAtStartPar
East Nusa Tenggara
&
\sphinxAtStartPar
IDNNT
\\
\hline
\sphinxAtStartPar
Papua
&
\sphinxAtStartPar
Papua
&
\sphinxAtStartPar
Papua
&
\sphinxAtStartPar
IDNPA
\\
\hline
\sphinxAtStartPar
Papua
&
\sphinxAtStartPar
Papua Barat
&
\sphinxAtStartPar
West Papua
&
\sphinxAtStartPar
IDNPB
\\
\hline
\sphinxAtStartPar
Sulawesi
&
\sphinxAtStartPar
Gorontalo
&
\sphinxAtStartPar
Gorontalo
&
\sphinxAtStartPar
IDNGO
\\
\hline
\sphinxAtStartPar
Sulawesi
&
\sphinxAtStartPar
Sulawesi Selatan
&
\sphinxAtStartPar
South Sulawesi
&
\sphinxAtStartPar
IDNSN
\\
\hline
\sphinxAtStartPar
Sulawesi
&
\sphinxAtStartPar
Sulawesi Tengah
&
\sphinxAtStartPar
Central Sulawesi
&
\sphinxAtStartPar
IDNST
\\
\hline
\sphinxAtStartPar
Sulawesi
&
\sphinxAtStartPar
Sulawesi Tenggara
&
\sphinxAtStartPar
Southeast Sulawesi
&
\sphinxAtStartPar
IDNSG
\\
\hline
\sphinxAtStartPar
Sulawesi
&
\sphinxAtStartPar
Sulawesi Utara
&
\sphinxAtStartPar
North Sulawesi
&
\sphinxAtStartPar
IDNSA
\\
\hline
\sphinxAtStartPar
Sulawesi
&
\sphinxAtStartPar
Sulawesi Barat
&
\sphinxAtStartPar
West Sulawesi
&
\sphinxAtStartPar
IDNSR
\\
\hline
\sphinxAtStartPar
Sumatera
&
\sphinxAtStartPar
Aceh
&
\sphinxAtStartPar
Aceh
&
\sphinxAtStartPar
IDNAC
\\
\hline
\sphinxAtStartPar
Sumatera
&
\sphinxAtStartPar
Bengkulu
&
\sphinxAtStartPar
Bengkulu
&
\sphinxAtStartPar
IDNBE
\\
\hline
\sphinxAtStartPar
Sumatera
&
\sphinxAtStartPar
Jambi
&
\sphinxAtStartPar
Jambi
&
\sphinxAtStartPar
IDNJA
\\
\hline
\sphinxAtStartPar
Sumatera
&
\sphinxAtStartPar
Kepulauan Bangka Belitung
&
\sphinxAtStartPar
Bangka\sphinxhyphen{}Belitung
&
\sphinxAtStartPar
IDNBB
\\
\hline
\sphinxAtStartPar
Sumatera
&
\sphinxAtStartPar
Kepulauan Riau
&
\sphinxAtStartPar
Riau Islands
&
\sphinxAtStartPar
IDNKR
\\
\hline
\sphinxAtStartPar
Sumatera
&
\sphinxAtStartPar
Lampung
&
\sphinxAtStartPar
Lampung
&
\sphinxAtStartPar
IDNLA
\\
\hline
\sphinxAtStartPar
Sumatera
&
\sphinxAtStartPar
Riau
&
\sphinxAtStartPar
Riau
&
\sphinxAtStartPar
IDNRI
\\
\hline
\sphinxAtStartPar
Sumatera
&
\sphinxAtStartPar
Sumatera Barat
&
\sphinxAtStartPar
West Sumatra
&
\sphinxAtStartPar
IDNSB
\\
\hline
\sphinxAtStartPar
Sumatera
&
\sphinxAtStartPar
Sumatera Selatan
&
\sphinxAtStartPar
South Sumatra
&
\sphinxAtStartPar
IDNSS
\\
\hline
\sphinxAtStartPar
Sumatera
&
\sphinxAtStartPar
Sumatera Utara
&
\sphinxAtStartPar
North Sumatra
&
\sphinxAtStartPar
IDNSU
\\
\hline
\end{longtable}\sphinxatlongtableend\end{savenotes}

\noindent\sphinxincludegraphics{{Indonesia_provinces_english}.png}


\subsubsection{Temporal resolution}
\label{\detokenize{1_indonesia:temporal-resolution}}
\sphinxAtStartPar
Each year is divided into 6 ‘Seasons’ {[}S1\sphinxhyphen{}S6{]}:


\begin{savenotes}\sphinxattablestart
\centering
\sphinxcapstartof{table}
\sphinxthecaptionisattop
\sphinxcaption{Teporal resolution \sphinxhyphen{} Seasons}\label{\detokenize{1_indonesia:id2}}
\sphinxaftertopcaption
\begin{tabular}[t]{|\X{50}{100}|\X{50}{100}|}
\hline
\sphinxstyletheadfamily 
\sphinxAtStartPar
Season
&\sphinxstyletheadfamily 
\sphinxAtStartPar
Months
\\
\hline
\sphinxAtStartPar
S1
&
\sphinxAtStartPar
Jan, Feb
\\
\hline
\sphinxAtStartPar
S2
&
\sphinxAtStartPar
Mar, Apr
\\
\hline
\sphinxAtStartPar
S3
&
\sphinxAtStartPar
May, Jun
\\
\hline
\sphinxAtStartPar
S4
&
\sphinxAtStartPar
Jul, Aug
\\
\hline
\sphinxAtStartPar
S5
&
\sphinxAtStartPar
Sep, Oct
\\
\hline
\sphinxAtStartPar
S6
&
\sphinxAtStartPar
Nov, Dec
\\
\hline
\end{tabular}
\par
\sphinxattableend\end{savenotes}

\sphinxAtStartPar
Each ‘Season’ is further divided into 12 ‘Daily Time Brackets’:


\begin{savenotes}\sphinxattablestart
\centering
\sphinxcapstartof{table}
\sphinxthecaptionisattop
\sphinxcaption{Temporal resolution \sphinxhyphen{} daily time brackets}\label{\detokenize{1_indonesia:id3}}
\sphinxaftertopcaption
\begin{tabular}[t]{|\X{50}{100}|\X{50}{100}|}
\hline
\sphinxstyletheadfamily 
\sphinxAtStartPar
Daily time bracket
&\sphinxstyletheadfamily 
\sphinxAtStartPar
Hours of the day
\\
\hline
\sphinxAtStartPar
D1
&
\sphinxAtStartPar
0, 1
\\
\hline
\sphinxAtStartPar
D2
&
\sphinxAtStartPar
2, 3
\\
\hline
\sphinxAtStartPar
D3
&
\sphinxAtStartPar
4, 5
\\
\hline
\sphinxAtStartPar
D4
&
\sphinxAtStartPar
6, 7
\\
\hline
\sphinxAtStartPar
D5
&
\sphinxAtStartPar
8, 9
\\
\hline
\sphinxAtStartPar
D6
&
\sphinxAtStartPar
10, 11
\\
\hline
\sphinxAtStartPar
D7
&
\sphinxAtStartPar
12, 13
\\
\hline
\sphinxAtStartPar
D8
&
\sphinxAtStartPar
14, 15
\\
\hline
\sphinxAtStartPar
D9
&
\sphinxAtStartPar
16, 17
\\
\hline
\sphinxAtStartPar
D10
&
\sphinxAtStartPar
18, 19
\\
\hline
\sphinxAtStartPar
D11
&
\sphinxAtStartPar
20,  21
\\
\hline
\sphinxAtStartPar
D12
&
\sphinxAtStartPar
22,  23
\\
\hline
\end{tabular}
\par
\sphinxattableend\end{savenotes}

\sphinxAtStartPar
Together, there are 72 representative ‘timeslices’ in the model. The temporal
resolution is the same for the entire model period.


\subsubsection{Model horizon}
\label{\detokenize{1_indonesia:model-horizon}}
\sphinxAtStartPar
Base year \sphinxhyphen{} 2021

\sphinxAtStartPar
End year \sphinxhyphen{} 2050


\subsection{Key assumptions}
\label{\detokenize{1_indonesia:key-assumptions}}

\subsubsection{Discount rates}
\label{\detokenize{1_indonesia:discount-rates}}
\sphinxAtStartPar
The model includes two types of discount rates (DR): ‘social’ and ‘financial’.
The social DR is applied across the entire model and represents the relative
weighting of present and future costs and benefits. A low social DR weights the
present and the future more similarly than a high DR. The financial DR is
technology\sphinxhyphen{}specific and represents the weighted average cost of capital (WACC)
for a given technology (e.g. power plant). The model assumes a value of
\sphinxstylestrong{10\%} for both the social and financial discount rates. The latter is based
on the \sphinxhref{https://www.iea.org/data-and-statistics/data-tools/cost-of-capital-observatory}{IEA Cost of Capital Observatory}


\subsubsection{Reserve Margin}
\label{\detokenize{1_indonesia:reserve-margin}}\begin{itemize}
\item {} 
\sphinxAtStartPar
Reserve margin 60\% \sphinxhyphen{}\textgreater{} 35\%

\item {} 
\end{itemize}


\subsection{Data}
\label{\detokenize{1_indonesia:data}}

\subsubsection{Technology costs}
\label{\detokenize{1_indonesia:technology-costs}}
\sphinxAtStartPar
\sphinxhref{https://www.nrel.gov/docs/fy21osti/79236.pdf}{NREL}


\begin{savenotes}\sphinxattablestart
\centering
\sphinxcapstartof{table}
\sphinxthecaptionisattop
\sphinxcaption{Technology cost projections (Capital)}\label{\detokenize{1_indonesia:id4}}
\sphinxaftertopcaption
\begin{tabular}[t]{|\X{75}{325}|\X{50}{325}|\X{50}{325}|\X{50}{325}|\X{50}{325}|\X{50}{325}|}
\hline
\sphinxstyletheadfamily 
\sphinxAtStartPar
Technology
&\sphinxstyletheadfamily 
\sphinxAtStartPar
Unit
&\sphinxstyletheadfamily 
\sphinxAtStartPar
2020
&\sphinxstyletheadfamily 
\sphinxAtStartPar
2030
&\sphinxstyletheadfamily 
\sphinxAtStartPar
2040
&\sphinxstyletheadfamily 
\sphinxAtStartPar
2050
\\
\hline
\sphinxAtStartPar
BAT
&
\sphinxAtStartPar
USD$_{\text{2020}}$/kWh
&
\sphinxAtStartPar
91750
&
\sphinxAtStartPar
55000
&
\sphinxAtStartPar
48194
&
\sphinxAtStartPar
41389
\\
\hline
\sphinxAtStartPar
BIO
&
\sphinxAtStartPar
USD$_{\text{2020}}$/kW
&
\sphinxAtStartPar
2038
&
\sphinxAtStartPar
2038
&
\sphinxAtStartPar
2013
&
\sphinxAtStartPar
2013
\\
\hline
\sphinxAtStartPar
CCG
&
\sphinxAtStartPar
USD$_{\text{2020}}$/kW
&
\sphinxAtStartPar
790
&
\sphinxAtStartPar
790
&
\sphinxAtStartPar
790
&
\sphinxAtStartPar
790
\\
\hline
\sphinxAtStartPar
COA
&
\sphinxAtStartPar
USD$_{\text{2020}}$/kW
&
\sphinxAtStartPar
1442
&
\sphinxAtStartPar
1442
&
\sphinxAtStartPar
1442
&
\sphinxAtStartPar
1442
\\
\hline
\sphinxAtStartPar
COG
&
\sphinxAtStartPar
USD$_{\text{2020}}$/kW
&
\sphinxAtStartPar
1030
&
\sphinxAtStartPar
1030
&
\sphinxAtStartPar
1030
&
\sphinxAtStartPar
1030
\\
\hline
\sphinxAtStartPar
CSP
&
\sphinxAtStartPar
USD$_{\text{2020}}$/kW
&
\sphinxAtStartPar
3088
&
\sphinxAtStartPar
3088
&
\sphinxAtStartPar
2675
&
\sphinxAtStartPar
2675
\\
\hline
\sphinxAtStartPar
GEO
&
\sphinxAtStartPar
USD$_{\text{2020}}$/kW
&
\sphinxAtStartPar
2625
&
\sphinxAtStartPar
2625
&
\sphinxAtStartPar
2513
&
\sphinxAtStartPar
2513
\\
\hline
\sphinxAtStartPar
HYD
&
\sphinxAtStartPar
USD$_{\text{2020}}$/kW
&
\sphinxAtStartPar
2088
&
\sphinxAtStartPar
2088
&
\sphinxAtStartPar
2113
&
\sphinxAtStartPar
2113
\\
\hline
\sphinxAtStartPar
OCG
&
\sphinxAtStartPar
USD$_{\text{2020}}$/kW
&
\sphinxAtStartPar
425
&
\sphinxAtStartPar
425
&
\sphinxAtStartPar
425
&
\sphinxAtStartPar
425
\\
\hline
\sphinxAtStartPar
OIL
&
\sphinxAtStartPar
USD$_{\text{2020}}$/kW
&
\sphinxAtStartPar
1240
&
\sphinxAtStartPar
1240
&
\sphinxAtStartPar
1240
&
\sphinxAtStartPar
1240
\\
\hline
\sphinxAtStartPar
OTH
&
\sphinxAtStartPar
USD$_{\text{2020}}$/kW
&
\sphinxAtStartPar
1240
&
\sphinxAtStartPar
1240
&
\sphinxAtStartPar
1240
&
\sphinxAtStartPar
1240
\\
\hline
\sphinxAtStartPar
PET
&
\sphinxAtStartPar
USD$_{\text{2020}}$/kW
&
\sphinxAtStartPar
1240
&
\sphinxAtStartPar
1240
&
\sphinxAtStartPar
1240
&
\sphinxAtStartPar
1240
\\
\hline
\sphinxAtStartPar
SPV
&
\sphinxAtStartPar
USD$_{\text{2020}}$/kW
&
\sphinxAtStartPar
723
&
\sphinxAtStartPar
723
&
\sphinxAtStartPar
615
&
\sphinxAtStartPar
615
\\
\hline
\sphinxAtStartPar
URN
&
\sphinxAtStartPar
USD$_{\text{2020}}$/kW
&
\sphinxAtStartPar
3263
&
\sphinxAtStartPar
3263
&
\sphinxAtStartPar
3200
&
\sphinxAtStartPar
3200
\\
\hline
\sphinxAtStartPar
WAS
&
\sphinxAtStartPar
USD$_{\text{2020}}$/kW
&
\sphinxAtStartPar
6663
&
\sphinxAtStartPar
6663
&
\sphinxAtStartPar
6550
&
\sphinxAtStartPar
6550
\\
\hline
\sphinxAtStartPar
WAV
&
\sphinxAtStartPar
USD$_{\text{2020}}$/kW
&
\sphinxAtStartPar
4475
&
\sphinxAtStartPar
4475
&
\sphinxAtStartPar
3350
&
\sphinxAtStartPar
3350
\\
\hline
\sphinxAtStartPar
WOF
&
\sphinxAtStartPar
USD$_{\text{2020}}$/kW
&
\sphinxAtStartPar
2355
&
\sphinxAtStartPar
2355
&
\sphinxAtStartPar
1968
&
\sphinxAtStartPar
1968
\\
\hline
\sphinxAtStartPar
WON
&
\sphinxAtStartPar
USD$_{\text{2020}}$/kW
&
\sphinxAtStartPar
1513
&
\sphinxAtStartPar
1513
&
\sphinxAtStartPar
1475
&
\sphinxAtStartPar
1475
\\
\hline
\end{tabular}
\par
\sphinxattableend\end{savenotes}


\begin{savenotes}\sphinxattablestart
\centering
\sphinxcapstartof{table}
\sphinxthecaptionisattop
\sphinxcaption{Technology cost projections (Fixed)}\label{\detokenize{1_indonesia:id5}}
\sphinxaftertopcaption
\begin{tabular}[t]{|\X{75}{325}|\X{50}{325}|\X{50}{325}|\X{50}{325}|\X{50}{325}|\X{50}{325}|}
\hline
\sphinxstyletheadfamily 
\sphinxAtStartPar
Technology
&\sphinxstyletheadfamily 
\sphinxAtStartPar
Unit
&\sphinxstyletheadfamily 
\sphinxAtStartPar
2020
&\sphinxstyletheadfamily 
\sphinxAtStartPar
2030
&\sphinxstyletheadfamily 
\sphinxAtStartPar
2040
&\sphinxstyletheadfamily 
\sphinxAtStartPar
2050
\\
\hline
\sphinxAtStartPar
BIO
&
\sphinxAtStartPar
USD$_{\text{2020}}$/kW
&&&&\\
\hline
\sphinxAtStartPar
CCG
&
\sphinxAtStartPar
USD$_{\text{2020}}$/kW
&&&&\\
\hline
\sphinxAtStartPar
COA
&
\sphinxAtStartPar
USD$_{\text{2020}}$/kW
&&&&\\
\hline
\sphinxAtStartPar
COG
&
\sphinxAtStartPar
USD$_{\text{2020}}$/kW
&&&&\\
\hline
\sphinxAtStartPar
CSP
&
\sphinxAtStartPar
USD$_{\text{2020}}$/kW
&&&&\\
\hline
\sphinxAtStartPar
GEO
&
\sphinxAtStartPar
USD$_{\text{2020}}$/kW
&&&&\\
\hline
\sphinxAtStartPar
HYD
&
\sphinxAtStartPar
USD$_{\text{2020}}$/kW
&&&&\\
\hline
\sphinxAtStartPar
OCG
&
\sphinxAtStartPar
USD$_{\text{2020}}$/kW
&&&&\\
\hline
\sphinxAtStartPar
OIL
&
\sphinxAtStartPar
USD$_{\text{2020}}$/kW
&&&&\\
\hline
\sphinxAtStartPar
OTH
&
\sphinxAtStartPar
USD$_{\text{2020}}$/kW
&&&&\\
\hline
\sphinxAtStartPar
PET
&
\sphinxAtStartPar
USD$_{\text{2020}}$/kW
&&&&\\
\hline
\sphinxAtStartPar
SPV
&
\sphinxAtStartPar
USD$_{\text{2020}}$/kW
&&&&\\
\hline
\sphinxAtStartPar
URN
&
\sphinxAtStartPar
USD$_{\text{2020}}$/kW
&&&&\\
\hline
\sphinxAtStartPar
WAS
&
\sphinxAtStartPar
USD$_{\text{2020}}$/kW
&&&&\\
\hline
\sphinxAtStartPar
WAV
&
\sphinxAtStartPar
USD$_{\text{2020}}$/kW
&&&&\\
\hline
\sphinxAtStartPar
WOF
&
\sphinxAtStartPar
USD$_{\text{2020}}$/kW
&&&&\\
\hline
\sphinxAtStartPar
WON
&
\sphinxAtStartPar
USD$_{\text{2020}}$/kW
&&&&\\
\hline
\end{tabular}
\par
\sphinxattableend\end{savenotes}


\subsubsection{Renewable Energy Profiles}
\label{\detokenize{1_indonesia:renewable-energy-profiles}}\begin{itemize}
\item {} 
\sphinxAtStartPar
renewables.ninja

\end{itemize}


\subsubsection{Renewable Energy Potentials}
\label{\detokenize{1_indonesia:renewable-energy-potentials}}\begin{itemize}
\item {} 
\sphinxAtStartPar
GEO: Volcanostratigraphy of Batukuwung\sphinxhyphen{}Parakasak Geothermal Area, Serang Regency, West Java.

\item {} 
\sphinxAtStartPar
SPV, WON, HYD: Beyond 443 GW

\item {} 
\sphinxAtStartPar
HYD: \sphinxurl{https://www.hydropower.org/blog/indonesia-promotes-hydropower-to-create-the-demand-for-industrial-development\#:~:text=The\%20biggest\%20hydropower\%20potential\%20is,Tenggara\%2DMaluku\%20is\%201.1\%20GW}.

\end{itemize}


\begin{savenotes}\sphinxatlongtablestart\begin{longtable}[c]{|\X{75}{325}|\X{50}{325}|\X{50}{325}|\X{50}{325}|\X{50}{325}|\X{50}{325}|}
\sphinxthelongtablecaptionisattop
\caption{Renewable energy potential\strut}\label{\detokenize{1_indonesia:id6}}\\*[\sphinxlongtablecapskipadjust]
\hline
\sphinxstyletheadfamily 
\sphinxAtStartPar
Potential (GW) by province
&\sphinxstyletheadfamily 
\sphinxAtStartPar
Solar photovoltaic
&\sphinxstyletheadfamily 
\sphinxAtStartPar
Wind \sphinxhyphen{} Onshore
&\sphinxstyletheadfamily 
\sphinxAtStartPar
Hydropower
&\sphinxstyletheadfamily 
\sphinxAtStartPar
Biomass
&\sphinxstyletheadfamily 
\sphinxAtStartPar
Geothermal
\\
\hline
\endfirsthead

\multicolumn{6}{c}%
{\makebox[0pt]{\sphinxtablecontinued{\tablename\ \thetable{} \textendash{} continued from previous page}}}\\
\hline
\sphinxstyletheadfamily 
\sphinxAtStartPar
Potential (GW) by province
&\sphinxstyletheadfamily 
\sphinxAtStartPar
Solar photovoltaic
&\sphinxstyletheadfamily 
\sphinxAtStartPar
Wind \sphinxhyphen{} Onshore
&\sphinxstyletheadfamily 
\sphinxAtStartPar
Hydropower
&\sphinxstyletheadfamily 
\sphinxAtStartPar
Biomass
&\sphinxstyletheadfamily 
\sphinxAtStartPar
Geothermal
\\
\hline
\endhead

\hline
\multicolumn{6}{r}{\makebox[0pt][r]{\sphinxtablecontinued{continues on next page}}}\\
\endfoot

\endlastfoot

\sphinxAtStartPar
Banten
&
\sphinxAtStartPar
175.94
&
\sphinxAtStartPar
1.21
&
\sphinxAtStartPar
4.67
&
\sphinxAtStartPar
0.93
&
\sphinxAtStartPar
1.31
\\
\hline
\sphinxAtStartPar
Jakarta Raya
&
\sphinxAtStartPar
11.60
&
\sphinxAtStartPar
0.02
&
\sphinxAtStartPar
0.09
&
\sphinxAtStartPar
0.02
&
\sphinxAtStartPar
0.03
\\
\hline
\sphinxAtStartPar
Jawa Barat
&
\sphinxAtStartPar
11.97
&
\sphinxAtStartPar
0.00
&
\sphinxAtStartPar
0.11
&
\sphinxAtStartPar
0.34
&
\sphinxAtStartPar
0.64
\\
\hline
\sphinxAtStartPar
Jawa Tengah
&
\sphinxAtStartPar
55.70
&
\sphinxAtStartPar
0.00
&
\sphinxAtStartPar
0.85
&
\sphinxAtStartPar
0.35
&
\sphinxAtStartPar
0.03
\\
\hline
\sphinxAtStartPar
Jawa Timur
&
\sphinxAtStartPar
11.97
&
\sphinxAtStartPar
0.01
&
\sphinxAtStartPar
1.01
&
\sphinxAtStartPar
0.02
&
\sphinxAtStartPar
0.04
\\
\hline
\sphinxAtStartPar
Yogyakarta
&
\sphinxAtStartPar
5.98
&
\sphinxAtStartPar
0.00
&
\sphinxAtStartPar
0.00
&
\sphinxAtStartPar
0.01
&
\sphinxAtStartPar
0.00
\\
\hline
\sphinxAtStartPar
Kalimantan Barat
&
\sphinxAtStartPar
249.47
&
\sphinxAtStartPar
0.00
&
\sphinxAtStartPar
0.65
&
\sphinxAtStartPar
2.06
&
\sphinxAtStartPar
1.05
\\
\hline
\sphinxAtStartPar
Kalimantan Selatan
&
\sphinxAtStartPar
40.91
&
\sphinxAtStartPar
0.42
&
\sphinxAtStartPar
1.45
&
\sphinxAtStartPar
0.36
&
\sphinxAtStartPar
6.63
\\
\hline
\sphinxAtStartPar
Kalimantan Tengah
&
\sphinxAtStartPar
54.23
&
\sphinxAtStartPar
0.19
&
\sphinxAtStartPar
1.27
&
\sphinxAtStartPar
0.11
&
\sphinxAtStartPar
1.83
\\
\hline
\sphinxAtStartPar
Kalimantan Timur
&
\sphinxAtStartPar
66.69
&
\sphinxAtStartPar
0.21
&
\sphinxAtStartPar
1.31
&
\sphinxAtStartPar
0.16
&
\sphinxAtStartPar
1.16
\\
\hline
\sphinxAtStartPar
Kalimantan Utara
&
\sphinxAtStartPar
983.49
&
\sphinxAtStartPar
0.00
&
\sphinxAtStartPar
6.54
&
\sphinxAtStartPar
3.00
&
\sphinxAtStartPar
0.05
\\
\hline
\sphinxAtStartPar
Maluku
&
\sphinxAtStartPar
193.80
&
\sphinxAtStartPar
0.09
&
\sphinxAtStartPar
0.93
&
\sphinxAtStartPar
0.87
&
\sphinxAtStartPar
0.00
\\
\hline
\sphinxAtStartPar
Maluku Utara
&
\sphinxAtStartPar
586.46
&
\sphinxAtStartPar
0.00
&
\sphinxAtStartPar
2.49
&
\sphinxAtStartPar
3.77
&
\sphinxAtStartPar
0.00
\\
\hline
\sphinxAtStartPar
Bali
&
\sphinxAtStartPar
1100.71
&
\sphinxAtStartPar
0.00
&
\sphinxAtStartPar
6.63
&
\sphinxAtStartPar
2.40
&
\sphinxAtStartPar
0.00
\\
\hline
\sphinxAtStartPar
Nusa Tenggara Barat
&
\sphinxAtStartPar
135.59
&
\sphinxAtStartPar
0.00
&
\sphinxAtStartPar
5.01
&
\sphinxAtStartPar
0.55
&
\sphinxAtStartPar
0.00
\\
\hline
\sphinxAtStartPar
Nusa Tenggara Timur
&
\sphinxAtStartPar
209.08
&
\sphinxAtStartPar
0.00
&
\sphinxAtStartPar
0.00
&
\sphinxAtStartPar
0.23
&
\sphinxAtStartPar
0.03
\\
\hline
\sphinxAtStartPar
Papua
&
\sphinxAtStartPar
20.49
&
\sphinxAtStartPar
0.00
&
\sphinxAtStartPar
0.00
&
\sphinxAtStartPar
0.66
&
\sphinxAtStartPar
0.00
\\
\hline
\sphinxAtStartPar
Papua Barat
&
\sphinxAtStartPar
64.98
&
\sphinxAtStartPar
0.00
&
\sphinxAtStartPar
0.47
&
\sphinxAtStartPar
1.65
&
\sphinxAtStartPar
2.86
\\
\hline
\sphinxAtStartPar
Gorontalo
&
\sphinxAtStartPar
196.59
&
\sphinxAtStartPar
4.86
&
\sphinxAtStartPar
0.55
&
\sphinxAtStartPar
0.06
&
\sphinxAtStartPar
0.23
\\
\hline
\sphinxAtStartPar
Sulawesi Selatan
&
\sphinxAtStartPar
81.05
&
\sphinxAtStartPar
0.00
&
\sphinxAtStartPar
0.18
&
\sphinxAtStartPar
0.01
&
\sphinxAtStartPar
0.23
\\
\hline
\sphinxAtStartPar
Sulawesi Tengah
&
\sphinxAtStartPar
40.58
&
\sphinxAtStartPar
0.03
&
\sphinxAtStartPar
0.03
&
\sphinxAtStartPar
0.07
&
\sphinxAtStartPar
0.14
\\
\hline
\sphinxAtStartPar
Sulawesi Tenggara
&
\sphinxAtStartPar
312.49
&
\sphinxAtStartPar
5.94
&
\sphinxAtStartPar
0.24
&
\sphinxAtStartPar
0.14
&
\sphinxAtStartPar
1.04
\\
\hline
\sphinxAtStartPar
Sulawesi Utara
&
\sphinxAtStartPar
571.50
&
\sphinxAtStartPar
0.16
&
\sphinxAtStartPar
19.98
&
\sphinxAtStartPar
0.36
&
\sphinxAtStartPar
0.03
\\
\hline
\sphinxAtStartPar
Sulawesi Barat
&
\sphinxAtStartPar
149.57
&
\sphinxAtStartPar
0.00
&
\sphinxAtStartPar
2.42
&
\sphinxAtStartPar
0.15
&
\sphinxAtStartPar
0.03
\\
\hline
\sphinxAtStartPar
Aceh
&
\sphinxAtStartPar
261.47
&
\sphinxAtStartPar
0.00
&
\sphinxAtStartPar
0.14
&
\sphinxAtStartPar
4.67
&
\sphinxAtStartPar
0.03
\\
\hline
\sphinxAtStartPar
Bengkulu
&
\sphinxAtStartPar
86.06
&
\sphinxAtStartPar
6.53
&
\sphinxAtStartPar
2.65
&
\sphinxAtStartPar
0.11
&
\sphinxAtStartPar
0.32
\\
\hline
\sphinxAtStartPar
Jambi
&
\sphinxAtStartPar
156.68
&
\sphinxAtStartPar
0.00
&
\sphinxAtStartPar
4.55
&
\sphinxAtStartPar
0.06
&
\sphinxAtStartPar
0.37
\\
\hline
\sphinxAtStartPar
Kepulauan Bangka Belitung
&
\sphinxAtStartPar
195.25
&
\sphinxAtStartPar
0.00
&
\sphinxAtStartPar
1.10
&
\sphinxAtStartPar
0.06
&
\sphinxAtStartPar
0.30
\\
\hline
\sphinxAtStartPar
Kepulauan Riau
&
\sphinxAtStartPar
14.04
&
\sphinxAtStartPar
0.00
&
\sphinxAtStartPar
0.00
&
\sphinxAtStartPar
0.00
&
\sphinxAtStartPar
0.87
\\
\hline
\sphinxAtStartPar
Lampung
&
\sphinxAtStartPar
72.76
&
\sphinxAtStartPar
0.00
&
\sphinxAtStartPar
2.87
&
\sphinxAtStartPar
1.00
&
\sphinxAtStartPar
1.62
\\
\hline
\sphinxAtStartPar
Riau
&
\sphinxAtStartPar
389.52
&
\sphinxAtStartPar
0.00
&
\sphinxAtStartPar
0.97
&
\sphinxAtStartPar
5.00
&
\sphinxAtStartPar
1.91
\\
\hline
\sphinxAtStartPar
Sumatera Barat
&
\sphinxAtStartPar
213.06
&
\sphinxAtStartPar
0.04
&
\sphinxAtStartPar
4.98
&
\sphinxAtStartPar
1.43
&
\sphinxAtStartPar
3.63
\\
\hline
\sphinxAtStartPar
Sumatera Selatan
&
\sphinxAtStartPar
7.35
&
\sphinxAtStartPar
0.00
&
\sphinxAtStartPar
0.06
&
\sphinxAtStartPar
0.01
&
\sphinxAtStartPar
0.01
\\
\hline
\sphinxAtStartPar
Sumatera Utara
&
\sphinxAtStartPar
22.27
&
\sphinxAtStartPar
0.00
&
\sphinxAtStartPar
0.89
&
\sphinxAtStartPar
0.11
&
\sphinxAtStartPar
0.00
\\
\hline
\end{longtable}\sphinxatlongtableend\end{savenotes}


\subsubsection{Energy demand projections}
\label{\detokenize{1_indonesia:energy-demand-projections}}\begin{itemize}
\item {} 
\sphinxAtStartPar
Own calculations

\end{itemize}


\subsubsection{Fuel Prices}
\label{\detokenize{1_indonesia:fuel-prices}}

\begin{savenotes}\sphinxattablestart
\centering
\sphinxcapstartof{table}
\sphinxthecaptionisattop
\sphinxcaption{Fuel price projections}\label{\detokenize{1_indonesia:id7}}
\sphinxaftertopcaption
\begin{tabular}[t]{|\X{75}{350}|\X{75}{350}|\X{50}{350}|\X{50}{350}|\X{50}{350}|\X{50}{350}|}
\hline
\sphinxstyletheadfamily 
\sphinxAtStartPar
Fuel
&\sphinxstyletheadfamily 
\sphinxAtStartPar
Unit
&\sphinxstyletheadfamily 
\sphinxAtStartPar
2020
&\sphinxstyletheadfamily 
\sphinxAtStartPar
2030
&\sphinxstyletheadfamily 
\sphinxAtStartPar
2040
&\sphinxstyletheadfamily 
\sphinxAtStartPar
2050
\\
\hline
\sphinxAtStartPar
Coal
&
\sphinxAtStartPar
USD$_{\text{2020}}$/mt
&
\sphinxAtStartPar
60.8
&
\sphinxAtStartPar
194.24
&
\sphinxAtStartPar
194.24
&
\sphinxAtStartPar
194.24
\\
\hline
\sphinxAtStartPar
Natural gas
&
\sphinxAtStartPar
USD$_{\text{2020}}$/mmbtu
&
\sphinxAtStartPar
3.2
&
\sphinxAtStartPar
23.86
&
\sphinxAtStartPar
23.86
&
\sphinxAtStartPar
23.86
\\
\hline
\sphinxAtStartPar
Oil
&
\sphinxAtStartPar
USD$_{\text{2020}}$/bbl
&
\sphinxAtStartPar
42.3
&
\sphinxAtStartPar
76.94
&
\sphinxAtStartPar
76.94
&
\sphinxAtStartPar
76.94
\\
\hline
\end{tabular}
\par
\sphinxattableend\end{savenotes}


\subsubsection{Electricity interconnectors}
\label{\detokenize{1_indonesia:electricity-interconnectors}}

\subsection{Scenarios}
\label{\detokenize{1_indonesia:scenarios}}
\sphinxAtStartPar
The model was used to explore three scenarios: \sphinxstyleemphasis{Current Policies {[}CP{]}},
\sphinxstyleemphasis{Least\sphinxhyphen{}cost {[}LC{]}}, and \sphinxstyleemphasis{Net\sphinxhyphen{}Zero {[}NZ{]}}. The scenarios represent alternate
pathways for the expansion of Indonesia’s electricity system. Each scenario
consists of a set of assumptions and constraints, as detailed below:


\subsubsection{Current policies}
\label{\detokenize{1_indonesia:current-policies}}
\sphinxAtStartPar
This scenario includes all implemented policies related to the expansion of
Indonesia’s electricity system as well as power plants under construction.
The policies included are:

\sphinxAtStartPar
And the future power plants included are:


\subsubsection{Least\sphinxhyphen{}cost}
\label{\detokenize{1_indonesia:least-cost}}

\subsubsection{Net\sphinxhyphen{}zero}
\label{\detokenize{1_indonesia:net-zero}}

\subsection{Results}
\label{\detokenize{1_indonesia:results}}

\subsubsection{Capacity expansion}
\label{\detokenize{1_indonesia:capacity-expansion}}



\subsubsection{Annual electricity generation mix}
\label{\detokenize{1_indonesia:annual-electricity-generation-mix}}



\subsubsection{Hourly electricity generation mix}
\label{\detokenize{1_indonesia:hourly-electricity-generation-mix}}



\subsection{Planned improvements}
\label{\detokenize{1_indonesia:planned-improvements}}\begin{itemize}
\item {} 
\sphinxAtStartPar
Interconnector expansion plans

\item {} 
\sphinxAtStartPar
Fossil fuel price projections

\item {} 
\sphinxAtStartPar
Plant\sphinxhyphen{}specific efficiencies

\item {} 
\sphinxAtStartPar
Hydropower capacity factor by plant / node

\item {} 
\sphinxAtStartPar
Technology\sphinxhyphen{}specific discount rates

\end{itemize}


\subsection{Model code, data, and workflow}
\label{\detokenize{1_indonesia:model-code-data-and-workflow}}
\sphinxAtStartPar
The entire workflow of FEO Global is available under an open license
at \sphinxtitleref{transition\sphinxhyphen{}zero/feo\sphinxhyphen{}esmod\sphinxhyphen{}osemosys}.
In addition, it uses only publicly available data and open source solver (CBC).


\subsection{References}
\label{\detokenize{1_indonesia:references}}\begin{itemize}
\item {} 
\sphinxAtStartPar
\sphinxhref{iea\_nze}{‘An Energy Sector Roadmap to Net Zero Emissions in Indonesia’, IEA, 2022}

\item {} 
\sphinxAtStartPar
\sphinxhref{https://www.irena.org/-/media/Files/IRENA/Agency/Publication/2022/Oct/IRENA\_Indonesia\_energy\_transition\_outlook\_2022.pdf?rev=b122956e990f485994b9e9d7075f696c}{‘Indonesia Energy Transition Outlook’, IRENA, 2022}

\item {} 
\sphinxAtStartPar
\sphinxhref{https://web.pln.co.id/statics/uploads/2021/10/ruptl-2021-2030.pdf}{‘RUPTL 2021\sphinxhyphen{}2030’, PLN, 2021}

\item {} 
\sphinxAtStartPar
\sphinxhref{https://iesr.or.id/en/pustaka/beyond-443-gw-indonesias-infinite-renewable-energy-potentials}{‘Beyond 443 GW \sphinxhyphen{} Indonesia’s Infinite Renewable Energy Potentials’, IESR, 2021}

\item {} 
\sphinxAtStartPar
\sphinxhref{https://unfccc.int/sites/default/files/resource/Indonesia\_LTS-LCCR\_2021.pdf}{‘Indonesia Long\sphinxhyphen{}Term Strategy for Low Carbon and Climate Resilience 2050’}

\end{itemize}





\renewcommand{\indexname}{Index}
\printindex
\end{document}